	\documentclass[Report.tex]{subfiles}
\begin{document}
In modern day society, data is collected at a rapidly increasing rate in all fields, and it has become a necessity to present data in different ways in order for humans to make sense of it. One way of doing so is through data visualization. Visualization of data can help humans' understanding of large data sets, as the data can be summarized very effectively, and patterns can quickly be recognized. When making visualizations, there are several considerations the designer must make, including: What are the semantics of the data being visualized, what is the target of the visualization and how the visualization is going to be created. It is also important to consider the human cognitive system when designing visualizations, such that the visualizations are designed to make it easier for humans to understand the data. Examples of this include limiting the amount of variables in the visualizations, as well as using contrasting colours. Given that visualizations are created to simplify large and complex data sets, sacrifices must be made to allow efficient interpretation. Too complex visualizations will not accomplish the task of presenting the data in an easily understandable manner, as humans will not be able to process all the information.
\\

In this project we have been working with visualization of football data. The data at our disposal for the project was provided by Prozone \footnote{http://prozonesports.stats.com/}. Prozone is a company that specializes in collecting and visualizing football data. They provided comprehensive data from the Danish Superliga, such as team stats like points, shots, goals and individual player stats.
\\

The visualization techniques and idioms presented in this report can be applied in many different settings. Some of the visualizations are very specific to football, for instance the ones with a football field, but these can easily be adapted to visualize data from other branches of sport. The more generic ones, such as the football parameter bubbles and the radar plots can easily be used to present other kinds of data. This could be financial data, biological data etc. 

The visualizations we have made can specifically be used by a team to analyze an opponent's playing style and adapt their strategy. The visualizations can also be used on websites, enabling regular football fans to view and interact with them. Furthermore, the visualizations can be used by betting companies and during live transmissions of football matches on TV.
\\

In order to design good visualizations, we will apply principles from the field of data science to present football data. We will use tools such as the programming language \emph{R} to process data and plot static visualizations. To make interactive and dynamic visualizations we will use the \emph{JavaScript} library \emph{D3}. 
Specifically, we will do this both by making visualizations that can help exploring the questions that we present below, and by doing exploratory analysis to discover new patterns. The specific questions that we will be investigating are: 
\begin{itemize}
\item How does a team evolve throughout a season?
\item How does a team’s playing style change throughout a match?
\item How does a winning team differ from a losing team?
\end{itemize}
During the visualization process we will use principles and analysis tools given by Tamara Munzner in “Visualization Analysis and Design” to make sure that the data is presented in an accurate and easily understandable manner. We will discuss the results of the visualizations and what we can learn from them.
\\

In this report we will start with an introduction of relevant theory, then a presentation of the data we have from Prozone, followed by analysis and visualizations. The design of each visualization will be explained as well as the findings. The findings and the visualizations will then be discussed, and followed by a conclusion.
\end{document}