\documentclass[Report.tex]{subfiles}
\begin{document}

\begin{figure}
\center
\includegraphics[scale=0.84]{"Shots_hf".pdf}
\caption{Grouped bar chart showing the ratio of headers and shots with foot as
a percentage of total shots, together with the percentage of chances that
results in goal}
\label{Fig:Shots_HF} 
\end{figure}

\paragraph{What-why-how\\}
This idiom (Figure \ref{Fig:Shots_HF}) shows the number of shots taken with the players' feet,
versus their head, as well as the ratio of goals per chance ("successful
chances"). The groups (teams) are ordered based on their ranking in the league,
with the highest ranking team as the first group. 
This visualization was created to see if
there is a correlation between a team's ranking, and how many shots were headers,
as well as how many of their chances were successful.

As with the other grouped bar charts, R was used exclusively, with the Excel
sheet as input data. 
The variables in question were selected, then split into different tables for
each team. Then the variables were summed, and shots from head and foot were
divided by total shots. The successful chances were calculated by dividing the
total number of goals with the total number of chances. 
ggplot2 was utilized to create the grouped bar chart, ordering the groups by the
ranking of the team, with the highest ranked team as the first group.

\paragraph{Findings\\}
There seems to be a linear correlation between the number of shots taken from outside the box
and the ranking of a team. Higher ranked teams tend to shoot more from inside the box, with AaB, Randers and OB being obvious outliers. 

\end{document}
