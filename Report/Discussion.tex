\documentclass[Report.tex]{subfiles}
\begin{document} 

% Playstyle change during match %
Our visualizations allowed us to find certain patterns in the data and partly prove or disprove some of our hypotheses. One of the things we intended to look into was how the playing style of a team changed throughout a match and to this end we saw some tendencies which seemed to apply. The first one was the fact that late in a match, it seemed as a general tendency that teams had more goal attempts than early on in the match. We actually found a direct correlation between time passed and how many goal attempts were taken which may imply that the players either become more desperate and shoot more, or the defense-line simply perform worse as the game progresses. Also, it seemed that teams would actually fail more passes at the very beginning of a match and at the start of the second half, which may be due to the fact they  have to get "warmed up" and thus make some mistakes at the beginning of each half.
%EDIT THIS PART%
These theories are based on the results of summing up all touch-related data throughout an entire season and thus there is a large amount of data to back up the notion. By looking at leagues from other countries as well as multiple seasons, it would be possible to either confirm or deny the hypothesis. 
%END EDIT PART%
Another thing which was found, though more based on single instances, was that a winning team would play more horizontally along the sides of a football field, whilst when losing they would play vertically and close to the middle instead. This was deducted by analyzing three matches between FCM and Viborg which indicated that when a top-tier team as FCM played against a supposedly on paper worse team like Viborg, they would in their supposed state of winning play more along the sides whilst when losing would put more focus in the center. Also, the rate at which FCM failed their passes when losing seemed to be rather high, when comparing with winning or tying. By cross checking with a lost match against another top-tier team, namely FCK we found that this rate actually dropped dramatically, even though the outcome of the matches were much alike. Thus, it seems that when a team, or at least FCM, are losing to a much worse team, they make more mistakes when passing. The cause can indeed be discussed, a possible explanation could be that they are stressing more to get ahead as they are supposed to win the match. However, it may as well simply be due to the fact that they are playing way under their normal level on the given day in the first place. 
% Winning VS Losing %
In regards to finding key differences between top-tier and low-tier teams, it turned out rather complicated to find any consistently different attributes. Some of the factors that we focused on, were based on summing some general player stats for all teams and compare them. Possibly counterintuitively, we were not able to find any direct correlation between how the best teams played nor the worse in opposition to each other, which lead to the notion that there may not be a single player attribute which leads to a winning team but rather a combination which is far more complicated. Another thing which was investigated was the distribution of players in terms of their nationality. We were not able to find any correlation between how many foreign players were on a team and their ranking, but we saw small tendencies of teams having more foreigners depending on their establishment, meaning teams that have been in the league for a while and thus have a larger budget, tended to have more foreigners which may simply be due to the fact that they have the money and thus the option to purchase players from other countries. \\
With regards to a team's evolution throughout a season, we were hoping, among other things, to find some correlation between how far in the season a game was, and how many fouls a team committed.
This was with the mindset that a team would get more aggressive as the end of the season came closer. However, how far into the season a match was, didn't seem to affect how many fouls were committed, instead, it kept jumping up and down throughout the season. This could indicate that the aggressiveness of a team's playing style is more dependent on their opponent, than the position and/or time in the league. There was however a small spike towards the last couple of matches, but nothing bigger than some of the spikes from mid-season. Something else that \ref{fig:FOULS} did show, was that the spikes in fouls committed in a match was consistently higher when away. This could indicate that a team tend to be more aggressive when away, than home. Another possibility could be that the referees tend to go easier on a home team. However, to validate that even in the slightest, significantly more in-depth analysis would have to take place. \\

We dove further into investigating how a team evolves throughout a season by looking at FCK. Here, the biggest takeaway comes when looking at the amount of points FCK has after each round. This visualization helped illustrate the highs and lows of the team’s season, and it showed how the team got off to a relatively slow start, only to get going and never looking back. We assumed that by looking at goal scoring and conceding in the first 10 games and comparing them to the remaining games, we could illustrate the evolution in the team. However, the results are simply far too inconsistent to reach any conclusions of note. 

Despite using PCA and manually testing different variables that would make sense to change throughout a season, there simply weren't enough season-wide data to draw any substantial conclusions regarding how a team evolves throughout a season.
% More win vs lose %

In order to try and establish which variables determine the differences between
winning teams and losing teams, a series of grouped bar charts were created, see
figures 11-15. \ref{fig:Goal_plot} shows different types of attacks that lead to a goal,
in percentages, grouped in teams by ranking order. For instance, Viborg scores
as often from set plays as on quick attacks.
On this chart, it is hard to notice any obvious tendencies between the higher
ranked teams to the lowest, but there are a couple of interesting things to
point out. FC Midtjylland, which is one of the top-tier teams in the league, is
the team that scores most goals on set plays, compared to total goals scored,
closely followed by AGF, which is currently the third-worst performing team
this season. It seems as such, that instead of finding a grouping of these
variables, we see them spread out across the teams, disregarding the ranking
somewhat. FC Midtjylland scores more on quick attacks than AGF, so one is left
to wonder if it is the combination of set plays and quick attacks that is more
effective, or if other variables come into play, which is probably the most
likely. 

It does seem like there is a slight tendency towards better teams scoring less
from possession based attacks, although there are some strong outliers, namely
AaB, OB and FC Nordsjaelland, who all score more than 60 \% of their
goals from possession based attacks, but these outliers all lie somewhat in the
middle of the league, with FC Nordsjaelland scoring much less on set plays
compared to the other two teams. 

Goals from open play, not attacking, are somewhat rare, as it is not often a
team scores suddenly, without being the attackers. The same goes for penalties,
although they are less rare than the aforementioned. For penalties, there does
seem to be a slight tendency for lower ranked teams to score more on these, and
this does make sense, as one could expect a lower ranked team to struggle with
scoring enough goals, and so penalties account for a larger share of their
goals. There are of course outliers, again FC Nordsjaelland, who didn't receive
any goals from penalties, but instead a few goals from non-attacking open play. 

\ref{fig:Pass_Direction} shows which direction the 12 different teams prefer to pass in,
divided between backward, sideways and forward passes. For the 2nd to the 4th
team, it seems like there is a local negative linear correlation between forward
and sideways passes, with forward passes being less for the 4th team, who pass
sideways much more frequently compared to team 2 and 3. The same is seen for the
last 4 teams, but in reverse order - teams pass less sideways and somewhat less
backwards, but more frequently forward. 

\ref{fig:Pass_Length} shows how far the teams pass on average, divided into three
categories - <17m, 17-34m, and >34m. 
On this chart, it is clear that FC Copenhagen less frequently passes the ball
long than the other top-tier teams, in fact there are only 3 other teams in the
league that pass less long balls than FC Copenhagen, and none of these are
top-tier teams. It is interesting then, to see that SonderjyskE is one of the
teams that pass long balls most frequently, being the team in 2nd place. When
you take into account that SonderjyskE was also among the teams passing the ball
forward most frequently, this makes good sense - it is better to pass short
balls backwards and long balls forward - a long ball backward would cost the
team momentum. Hobro is the other team that looks like SonderjyskE in respect to
both length and direction of passes. 

\ref{fig:Shots_HF} shows how many shots are taken with the foot versus head, as
percentages out of 100, as well as successful chances, which is calculated as
the total number of goals divided by the total number of chances, and selected
to indicate how successful a team is when trying to score. 
It clearly shows, counterintuitively, that the top ranking team, FC Copenhagen,
actually has the worst goal to chance ratio, which indicates that they are
maximizing the number of chances, and not necessarily the quality of these. This
team is also the most heading team in the league, leading by a relatively large
margin. 

SonderjyskE and FC Midtjylland stand out in terms of successful chances,
indicating they take the opposite approach to quantity vs quality, compared to
FC Copenhagen. Apart from these two teams, it looks like there is a slight
correlation between lower ranked teams having a higher ratio of successful
chances, with a bit of variance. This is well within reason, since lower ranked
teams create less chances, so they want to create the best possible chances in
general. It also seems as though lower ranked teams shoot more with their feet
than with their heads, but this is less clear, since the lowest 3 teams head
about as much as the 2nd and 3rd teams. 

\ref{fig:Shots_IO} shows where shots are taken from, divided into shots from inside the
box, and shots from outside the box. Successful chances is based on the same
data as in \ref{fig:Shots_HF}, and calculated the same way. 

From \ref{fig:Shots_IO} it is clear that the highest ranking team is also the team that
shoots from inside the box most frequently. Based on the earlier charts, one can
speculate that FC Copenhagen concentrates on passing the ball forward with short
passes, getting it inside the box, and then getting the ball into the goal,
instead of trying to create chances by lurking outside the box. 

As with the other charts, it looks - apart from SonderjyskE and FC Midtjylland -
as if there is somewhat of a linear correlation between different variables and
the ranking of teams. Here, it seems as if lower ranked teams tend to shoot more
often from outside the box, but again the correlation is too weak to conclude
anything meaningful. 

	
\end{document}
