\documentclass[Report.tex]{subfiles}
\begin{document}

\begin{figure}
\center
\includegraphics[scale=0.84]{"Pass_Length".pdf}
\caption{Grouped bar chart showing average pass length per team}
\label{Fig:Pass_Length}
\end{figure}



\subsubsection{What-why-how}
Figure \ref{Fig:Pass_Length} represents a grouped bar chart that shows the length of passes of a team, on average,
divided into three categories - less than 17 meters, between 17 and 34 meters,
and passes greater than 34 meters. These three variables are grouped by teams,
and the groups are sorted, based on the team's ranking in the league.
The purpose of this idiom is to see if there is a correlation between the length
of a team's passes, and their rank in the league.
Summing the average passing lengths of a team results in 100\%. 

This chart was made completely in R, by first selecting the desired variables
from the Excel sheet, then splitting this data into 12 sets, one per team. 
The columns were then summed, then divided with the total number of passes, to
get the different percentages. This was done in a loop, by adding variables and
values to predefined lists, instead of iteratively adding data to a data frame. 
By creating a data frame of the lists after iterating,
a new data frame is not created after every iteration, thus increasing
efficiency.
The order of the groups is decided on the levels of the data frame, and
thus these are reordered, based on the ranking of the team in question.

\end{document}
