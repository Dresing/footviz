\documentclass[Report.tex]{subfiles}

\begin{document} 
As data is being collected at a continuously increasing rate, it has become a necessity to represent it in alternative manners for humans to make sense of it. One way of doing so is through visualization. Visualization is a powerful tool that augments the human capabilities, making it easier for us to interpret data. However, many pitfalls and bad design decisions exist, that may cause the visualizations to not perform as intended. In this project, we have worked with football data and tried to create new forms of visualizations, whilst also answering some common questions regarding how one can show the differences between a top-tier and bottom-tier team. We have also looked at what their playing style is like, both in single matches and seasonal. To aid in creating visualizations, we have been using technologies such as \emph{R} and \emph{D3} for preprocessing and visualizing. Throughout the course of the project, we have also been using a variety of methods to help finding patterns. 

We ended up with some visualizations, that aided in finding patterns in the data sets. One of the patterns that we found, was that more shots were taken towards the end of a game. The number of shots progressively increases throughout the entire match. Some other tendencies the visualizations showed, was that teams usually make more failed passes at the beginning of both halves of the match, which may imply that some psychological factors affect the players and how they play. 

Another visualization also showed us that when a good team faces a worse one and loses, one can actually see a difference in how many failed passes they do relative to when they win. Furthermore, it also indicated that more play would develop around the middle of the field, which may indicate a rather frustrated playing style, as the players are not using the field to its full extend. 

In general we found that the visualizations containing the football field were the most effective. 
\end{document}