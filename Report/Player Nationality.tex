\documentclass[Report.tex]{subfiles}

\begin{document}

\begin{figure}
\center
\includegraphics[scale=0.5]{"Nationality of the Players".pdf}
\caption{Sankey-Chart showing the distribution of danish vs. foreign players on the different teams}
\label{Fig:Nationality}
\end{figure}


\subsubsection{What-why-how}
This visualisation shows the nationality of the players on the different teams in the league. It also shows the distribution of the foreign players among the other countries. This is done by a sankey chart, where the teams are sources, the targets are either Denmark and Foreign. On the second level the source is Foreign and the target is the other countries. The idiom is shown in Figure \ref{Fig:Nationality}.

The actions of this idiom is to summarise the players nationality, to explore and compare the different teams to find similarities/differences between the teams in comparison to the standing in the table. 

This is done by an interactive sankey chart. The view is manipulated by highlighting when hovering over a link, where we also get the percentage of players in this connection. The raw data is reduced by only focusing on the nationality of the players. We use lines as connections between the 

\subsubsection{Code}



We can clearly see some difference between the teams, some teams like 

\end{document}