In order to establish which variables determine the differences between
winning and losing teams, a series of grouped bar charts were created, see
Figures \ref{Fig:Goal_plot}-\ref{Fig:Shots_IO}.
Generally, there are some slight tendencies found on these charts, mostly between the highest and lowest ranked teams, with the teams in the middle being the general outliers. 
FC Midtjylland is the team that scores the most goals from set plays, closely followed by AGF, the third-lowest ranked team in the league. As FCM scores more on quick attacks than AGF, it is possible that it is the combination of these two types of goals that is more effective. It seems as if higher ranked teams score less on possession based attacks, and more from set plays and quick attacks, with a couple of teams in the middle of the league being outliers with many possession-based goals. Lower-ranked teams also tend to score more on penalties. 

Lower and higher ranked teams tend to pass the ball forward the most, except FCK, who passes the ball backwards more than other teams, as well as doing most short passes. It makes sense that the teams in the middle of the league tend to pass the ball sideways the most, as one would expect them to be more closely matched with their opponents in average, leading to longer time between attacks, which means more sideways passes are needed in order to carry out an attack. It makes sense, as well, that FCK makes many short passes, as a long pass backwards would be detrimental to getting the ball towards the opposition's goal. 

There is a tendency for higher ranked teams to make more shots from inside the box. This leads to the conclusion that FCK, for example, focuses on making short passes towards the opposition's goal, to maximize the amount of shooting chances. FCK has a low goal to chance ratio, and while SonderjyskE and FC Midtjylland often score on their chances, the general tendency seems to be that lower ranked teams score on most of their chances. Lower ranked teams create less chances, so it seems strategically sound to maximize the quality of these, instead of the quantity. 

The higher ranked teams also seem to make more headers towards goal, which makes sense as one would expect the players of higher ranked teams to win more duels against the defenders, and thereby having a higher chance of headers.

Generally, SonderjyskE and FC Midtjylland behave somewhat like lower-ranked teams, except with regards to how they score. One is left to wonder if lower ranked teams would be able to climb considerably in the league, just by focusing less on possession based attacks, and more on set plays and quick attacks. 
