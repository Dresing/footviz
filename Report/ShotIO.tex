\documentclass[Report.tex]{subfiles}
\begin{document}

\begin{figure}
\center
\includegraphics[scale=0.84]{"Shots_io".pdf}
\caption{Grouped bar chart showing how many shots were taken from inside the
box and outside the box, as percentages of total shots, as well as the
percentage of chances that result in goal.}
\label{Fig:Shots_IO} 
\end{figure}

\paragraph{What-why-how\\}
This visualization (Figure \ref{Fig:Shots_IO} shows how many of a team's shots are taken from inside the
box and outside the box, as well as the percentage of chances that are successful.
It was created to explore possible correlations between a team's ranking, and
where the shots originate from. 
The bar charts are grouped, based on a team's ranking, with the highest ranking
team as the first group. 

The bar chart was created in almost the same way as the other grouped bar
charts, utilizing R for both data processing, and creating the graph (with
ggplot2.)
Data was selected from the Excel sheet, and then split into the different teams,
before summing the variables. The shots from inside/outside the box were then
divided by total number of shots to get the percentages. Successful chances were
calculated by dividing total number of chances with total number of goals.
As with the other bar charts, the data was processed in a loop, processing one
team at a time, appending the values to a list, and then creating a data frame
outside the loop, with data from all the teams. ggplot2 was then used on this
data frame to create the chart.

\paragraph{Findings\\}
FCK has the lowest rate of successful chances out of all the teams, even though they are 
the highest ranked team. On the contrary, SonderjyskE and FC Midtjylland have the highest rate of successful chances. It seems, therefore, that it is viable to concentrate on either the quantity of chances, or the quality of these. 
FCK is the team that most often shoots by heading, with the teams in the middle of the league
somewhat preferring shots with foot. 



\end{document}
