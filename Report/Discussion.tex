\documentclass[Report.tex]{subfiles}
\begin{document} 

% Playstyle change during match %
Our visualization allowed us to find curtain patterns in the data and partly prove or disprove some common conceptions. One of the things we intended to look into was how the play-style of a team changed through out a match and to this end we some tendencies which seemed to apply. The first one was the fact that late in a match, it seemed as a general tendency that teams had more goal attempts, we actually found a direct correlation between time passed and how many goal attempts were taken which may imply that the players either becomes more desperate and shoots more, or the defense-line simply preform worse as the game progresses. Also, it seemed that teams would actually make more failed passes at the very beginning of a match and at the start of the second half, which may be due to the fact they  have to get "warmed up" and thus makes some mistakes at the beginning of each half.
%EDIT THIS PART%
These theories are based on the results of summing up all touch-related data throughout an entire season and thus there is a large amount of data to back it up the notion. What could strengthen the hypothesis would be to look at other international leagues and maybe even multiple seasons to hopefully find the same tendency. If done successfully, it would improve the accuracy of the assertion.
%END EDIT PART%
Another thing which was found, though more based on single instances, was that a winning team would play more horizontally along the sides of a football field, whilst when losing they would play vertically and close to the middle instead. This was deducted by analyzing three matches between FCM and Viborg which indicated that when a top-tier team as FCM played against a supposedly on paper worse team like Viborg, they would in their supposed state of winning play more along the sides whilst when losing would put more focus in the center. Also, the rate at which FCM failed their passes seemed to be rather high in respect to when winning or playing draw. By cross checking with a lost match against another top-tier team, namely FCK we found that this rate actually dropped dramatically, even tough the outcome of the matches were much alike. Thus, it seems that when a team, or at least FCM, are losing to a much worse team, they make more mistakes in their passing. The cause can indeed be discussed, some suggestion could be that they are stressing more to get ahead as they are supposed to win the match. However, it may as well simply be due to the fact that they are playing way under their normal level on the given day in the first place. 
% Winning VS Losing %
In regards to finding key differences between top-tier and low-tier teams, it turned out rather complicated to find any consistently different attributes. Some of the factors which we researched were based on summing some general player stats for all teams and compare them. Possibly counterintuitive, we were not able to find any direct correlation between how the best teams played nor the worse in opposition to each other, which lead to the notion that their may not be a single player attribute which leads to a winning team but rather a combination which is far more complicated. Another thing which were investigated was the distribution of players in terms of their nationality. We were not able to find any correlation between how many foreign players were on a team and their ranking but we saw small tendencies of teams having more foreigners depending on their establishment, meaning that teams which have been in the league for a while and thus have larger budget, tended to have more foreigners which may simply be due to the fact that they have the money and thus the option to purchase players from other countries. \\
When it comes to a team's change throughout a season, we were among other things hoping to find some correlation between how far in the season a game was, and how many fouls a team committed. This was with the mindset that a team would get more aggressive as the end of the season came. However, how far into the season a match were, didn't seem to affect how many fouls were committed, instead, it kept jumping up and down throughout the season. This could indicate that how aggressive a team play is more dependent on their opponent, than the position and/or time in the league. There were however a small spike towards the last couple of matches, but nothing bigger than some of the spikes from mid-season.\\
Despite using PCA and manually testing different variables that would make sense to change throughout a season, there simply weren't enough season-wide data to draw any substantial conclusions regarding how a team evolve throughout a season.

	
\end{document}