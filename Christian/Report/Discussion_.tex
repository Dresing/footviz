
\documentclass[Report.tex]{subfiles}

\begin{document}
In order to try and establish which variables determine the differences between
winning teams and losing teams, a series of grouped bar charts were created, see
figures 11-15. Figure 11 shows different types of attacks that lead to a goal,
in percentages, grouped in teams by ranking order. For instance, Viborg scores
as often from set plays as on quick attacks.
On this chart, it is hard to notice any obvious tendencies between the higher
ranked teams to the lowest, but there are a couple of interesting things to
point out. FC Midtjylland, which is one of the top-tier teams in the league, is
the team that scores most goals on set plays, compared to total goals scored,
closely followed by AGF, which is currently the third-worst performing team
this season. It seems as such, that instead of finding a grouping of these
variables, we see them spread out across the teams, disregarding the ranking
somewhat. FC Midtjylland scores more on quick attacks than AGF, so one is left
to wonder if it is the combination of set plays and quick attacks that is more
effective, or if other variables come into play, which is probably the most
likely. 

It does seem like there is a slight tendency towards better teams scoring less
from possession based attacks, although there are some strong outliers, namely
AaB, OB and FC Nordsjaelland, who all score more than 60 \% of their
goals from possession based attacks, but these outliers all lie somewhat in the
middle of the league, with FC Nordsjaelland scoring much less on set plays
compared to the other two teams. 

Goals from open play, not attacking, are somewhat rare, as it is not often a
team scores suddenly, without being the attackers. The same goes for penalties,
although they are less rare than the aforementioned. For penalties, there does
seem to be a slight tendency for lower ranked teams to score more on these, and
this does make sense, as one could expect a lower ranked team to struggle with
scoring enough goals, and so penalties account for a larger share of their
goals. There are of course outliers, again FC Nordsjaelland, who didn't receive
any goals from penalties, but instead a few goals from non-attacking open play. 

Figure 12 shows which direction the 12 different teams prefer to pass in,
divided between backward, sideways and forward passes. For the 2nd to the 4th
team, it seems like there is a local negative linear correlation between forward
and sideways passes, with forward passes being less for the 4th team, who pass
sideways much more frequently compared to team 2 and 3. The same is seen for the
last 4 teams, but in reverse order - teams pass less sideways and somewhat less
backwards, but more frequently forward. 

Figure 13 shows how far the teams pass on average, divided into three
categories - <17m, 17-34m, and >34m. 
On this chart, it is clear that FC Kobenhavn less frequently passes the ball
long than the other top-tier teams, in fact there are only 3 other teams in the
league that pass less long balls than FC Kobenhavn, and none of these are
top-tier teams. It is interesting then, to see that SonderjyskE is one of the
teams that pass long balls most frequently, being the team in 2nd place. When
you take into account that SonderjyskE was also among the teams passing the ball
forward most frequently, this makes good sense - it is better to pass short
balls backwards and long balls forward - a long ball backward would cost the
team momentum. Hobro is the other team that looks like SonderjyskE in respect to
both length and direction of passes. 

Figure 14 shows how many shots are taken with the foot versus head, as
percentages out of 100, as well as successful chances, which is calculated as
the total number of goals divided by the total number of chances, and selected
to indicate how successful a team is when trying to score. 
It clearly shows, counterintuitively, that the top ranking team, FC Kobenhavn,
actually has the worst goal to chance ratio, which indicates that they are
maximizing the number of chances, and not necessarily the quality of ditto. This
team is also the most heading team in the league, leading by a relatively large
margin. 

SonderjyskE and FC Midtjylland stand out in terms of successful chances,
indicating they take the opposite approach to quantity vs quality, compared to
FC Kobenhavn. Apart from these two teams, it looks like there is a slight
correlation between lower ranked teams having a higher ratio of successful
chances, with a bit of variance. This is well within reason, since lower ranked
teams create less chances, so they want to create the best possible chances in
general. It also seems as though lower ranked teams shoot more with their feet
than with their heads, but this is less clear, since the lowest 3 teams head
about as much as the 2nd and 3rd teams. 

Figure 15 shows where shots are taken from, divided into shots from inside the
box, and shots from outside the box. Successful chances is based on the same
data as in Figure 14, and calculated the same way. 

From Figure 15 it is clear that the highest ranking team is also the team that
shoots from inside the box most frequently. Based on the earlier charts, one can
speculate that FC Kobenhavn concentrates on passing the ball forward with short
passes, getting it inside the box, and then getting the ball into the goal,
instead of trying to create chances by lurking outside the box. 

As with the other charts, it looks - apart from SonderjyskE and FC Midtjylland -
as if there is somewhat of a linear correlation between different variables and
the ranking of teams. Here, it seems as if lower ranked teams tend to shoot more
often from outside the box, but again the correlation is too weak to conclude
anything meaningful. 
 
\end{document}
