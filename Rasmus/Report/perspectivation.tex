%Name in report: "Perspectivation"

\documentclass[Report.tex]{subfiles}

\begin{document}
The visualization techniques and idioms we have presented in this report can be applied in many different settings. Some of the visualizations
are very specific to football, for example the ones with a football field, but these can easily be adapted to visualize data from other branches
of sports. The more generic ones we have made such as the variation of a bubble chart and the radar plots can easily be used to present other
kinds of data which is not necessarily from sport. This could be financial data, biological data etc.
The visualizations we have made can specifically be used by a team to analyze how other teams play
and adopt their strategy when they meet those teams, or to analyze in general how teams play throughout a match, and then see if they can learn
something which can improve their own strategy. The visualizations can also be used on websites where normal football fans can view them and see if they can produce any interesting hypotheses, which they can try to test the next time they watch a game in the tv. The visualizations can be used by betting companies such that they can give their users tools such that the users can find out how they want to place their bets. The visualizations can also be used during live transmissions of football games in the tv, for example by showing statistics about the match during the intermission. In the academic setting, some of the visualizations can be used by psychologists for example, to learn something about the mentality of players during a game, for example by using the field representation of shots, to see where a player is likely to shoot from under certain conditions.

\end{document}