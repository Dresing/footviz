\documentclass[Report.tex]{subfiles}

\begin{document} 
Some of the patterns that were found, seemed to indicate a tendency for bottom-tier teams to shoot more, especially from outside the box. Bottom-tier teams seem to focus more on the quality of chances, with higher ranked teams focusing more on the quantity. It also seems like it is more effective to focus on set plays and quick attacks, instead of possession based attacks.

Whilst looking at how a team's playing style changed throughout a match, we found a relatively clear indication that teams tend to shoot more frequently as the game progresses. The causality was not clear but one could suggest that teams either become more desperate as the match progresses, or the defense line cuts a bit more slack later into the match. It also seemed as if teams were at higher risk of making unsuccessful passes at the beginning of each half, indicating that some factors must apply that cause players to perform slightly worse, which may be attributed to factors such as pressure or uncertainty of the opponent. Though these conclusions were supported by a substantial amount of records, some other tendencies were also found that were slightly less supported. The top-tier teams seemed to change their playing style in matches against bottom-tier teams, but it depended heavily on what the outcome was. Especially factors, such as failed passes, seemed to truly change depending on whom the opponent was, as it would often be higher if the opponent was worse on paper, as opposed to playing against a better one. Another tendency we found, was that a losing team may start playing more towards the middle of the field when being behind, as opposed to being ahead in the match. Although this theory is very poorly supported, it may very well lay grounds for a new hypothesis which is yet to be proven.

In regards to our visualizations, it turned out that those utilizing a football field as a facet view provided most usable information. However, the bar charts combined also provided valuable information, even though the individual charts did not provide much on their own.
\end{document}