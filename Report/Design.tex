\documentclass[Report.tex]{subfiles}
\begin{document}
The following section explains the process of designing visualizations. The process follows the "What-Why-How"-structure. In the end of this section is a brief explanation of the considerations a designer makes during the process.

The first question a designer has to answer is: What. What are the semantics of the data, what type of data is provided, what dataset is the data presented in. This is important, as understanding the specifics of the provided data is necessary for designing the most efficient visualization for this data. An efficient visualization illustrates the data as simple and understandable as possible. The efficient visualization designs for geometry are drastically different from those of tables. An example of geometry is the location on a football field where a given action has taken place, like a successful tackle. Here, a smart visualization could be illustrating where on the football field the most successful tackles are made, using a football field as a coordinate system. The data type would be positions and items, describing what happened, and where. In a table, similar information could be stored. However, instead of positions, it would contain attributes. For instance, an item could be a specific player, and an attribute could be how many successful tackles he had during a game. A clever design would be a bar chart, illustrating the player’s efficiency (amount of successful tackles in games) compared to other players at his position. This would help a coach figure out who is the best player for certain situations where defending is key.\\
After the data has been identified by the designer, the next question that needs to be answered is: “Why?”. What are the ideas behind the specific visualization? What is the target of this visualization? Is the goal of the visualization to consume existing data or to produce new information to visualize? This phase can be split in two: actions and targets. Actions are mainly comprised of analyzing, searching and querying. Analyzing depends on whether the goal is to consume or produce data. Search describes finding the data that answers the questions that are being asked. Searching is dependent on whether the location of the data is known or unknown, and whether the target is known or unknown. There are four kinds of searching; lookup, where location and target is known; browse, where location is known and target is unknown; locate, where the target is known but location is unknown; and explore, where neither target nor location is known. After the search, the goal is to query the targets at one of three scopes: identify, compare or summarize. Identify refers to a single target, compare refers to multiple targets and summarize refers to the full set of possible targets.

The other part of the “Why?” phase is Targets. Here, the designer identifies what the goal of the visualization is. To locate trends or outliers in datasets, identify correlations, the shape of objects, etc. 

The final question is “How?”. How is this visualization going to be designed, what design choices should be made to accomplish the goal of the visualization? Here, the designer should also decide the specifics of the visualization. In short, this is where the designer maps out how to arrive at the final visualization. How to encode the data, how to map it in the visualization, how to manipulate the visualizations for better visual efficiency, how to facet the data and how to reduce the data. We will go into further detail in this later in the report.


\end{document}