\documentclass[Report.tex]{subfiles}
%\documentclass{article}
\begin{document}



\begin{figure}
\center
\includegraphics[scale=0.84]{"Pass_Direction".pdf}
\caption{Grouped bar chart showing average pass directions per team}
\label{Fig:Pass_Direction} 
\end{figure}

\subsubsection{What-why-how}
The purpose of this idiom (Figure \ref{Fig:Pass_Direction}) is to see if there is a correlation between the
direction of a team's passes on average, and their ranking in the league. 
This is done by creating a grouped bar chart, with a group per team. The bar
chart is ordered by the ranking of the team, with the highest ranking team as
the first group from the left.

The bar chart, as well as the data processing for the chart, was created in R.
This was done by first selecting the variables in question from the Excel sheet,
then splitting it into 12 tables (one per team). The variables were then
summed, and each summed variable divided by total number of passes, to get the
percentages of short, medium and long passes. This was done by creating
variable and value lists, adding elements to these lists in a loop, and then
creating a data frame. 
ggplot2 was then used to create the bar chart, in much the same was as the other
grouped bar charts. 

\end{document} 
