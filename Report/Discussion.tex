\documentclass[Report.tex]{subfiles}
\begin{document} 

% Playstyle change during match %
Our visualizations allowed us to find certain patterns in the data and partly prove or disprove some of our hypotheses. One of the things we intended to look into was how the playing style of a team changed throughout a match, and to this end we saw some tendencies. The first one was the fact that late in a match, it seemed as if teams had more goal attempts than early in the match. We actually found a direct correlation between time passed and how many goal attempts were taken which may imply that the players either become more desperate and shoot more, or the defense-line simply perform worse as the game progresses. Based on the box plot and the visualization of field events in intervals, it seems like players are more likely to make goal attempts closer to the goal towards the end of the match as well as being more grouped. Also, it seemed that teams would actually fail more passes at the start of each half, which may be due to the fact that they have to get "warmed up".
\\

Another thing that was found was that winning teams would play more horizontally along the sides of the field, but when losing they would play more vertical and closer to the middle instead. Also, the rate at which a top-tier team made bad passes when losing, seemed to be higher than when winning. A possible explanation could be that they are stressing more to get ahead as they are expected to win. However, it may as well simply be due to the fact that they are playing way under their normal level on the given day in the first place, possibly because they expected to win.\\\

% Winning VS Losing %
In regards to finding key differences between top-tier and low-tier teams, it turned out rather complicated to find any consistently different attributes. Some of the factors that we focused on, were based on summing general player stats for all teams and comparing them. Possibly counterintuitively, we were not able to find any direct correlation between how the better teams played nor the worse in opposition to one another. This lead to the idea that there may not be a single player attribute which leads to a winning team but rather a combination which is far more complex. Another thing that was investigated, was the distribution of players in terms of their nationality. We were not able to find any correlation between how many foreign players were on a team and their ranking. However, we saw a weak tendency of teams having more foreigners depending on their establishment. Established teams have been in the league for a while and thus have a larger budget, which may explain why they tended to have more foreigners. \\

In regards to a team's evolution throughout a season, we tried to find a correlation between how far in the season a match was, and how many fouls a team committed.
This was with the mindset that a team would get more aggressive as the end of the season came closer. However, how far into the season a match was, didn't seem to affect how many fouls were committed. Instead, it kept jumping up and down throughout the season. This could indicate that the aggressiveness of a team's playing style is more dependent on their opponent, than the position and/or time in the league. There was however a small spike towards the last couple of matches, but nothing bigger than some of the spikes from the mid-season. Something else that Figure \ref{Fig:FOULS} did show, was that the spikes in fouls committed in a match was consistently higher when playing away. This could indicate that a team tend to be more aggressive when away, than home. Another possibility could be that the referees tend to go easier on a home team.
\\

We dove further into investigating how a team evolves throughout a season by looking at FCK. Here, the biggest takeaway comes when looking at the amount of points FCK had after each round. Figure \ref{Fig:points} and Figure \ref{Fig:fck_goals} helped illustrate the highs and lows of the team’s season, and it showed how the team got off to a relatively slow start, catching up later on. We assumed that by looking at the number of goals scored and conceded in the first 10 games, we could illustrate the evolution of the team. However, the results were simply far too inconsistent to reach any conclusions. 
% More win vs lose %

In order to establish which variables determine the differences between
winning and losing teams, a series of grouped bar charts were created, see
Figures \ref{Fig:Goal_plot}-\ref{Fig:Shots_IO}.
Generally, there are some slight tendencies found on these charts, mostly between the highest and lowest ranked teams, with the teams in the middle being the general outliers. 
FC Midtjylland is the team that scores the most goals from set plays, closely followed by AGF, the third-lowest ranked team in the league. As FCM scores more on quick attacks than AGF, it is possible that it is the combination of these two types of goals that is more effective. It seems as if higher ranked teams score less on possession based attacks, and more from set plays and quick attacks, with a couple of teams in the middle of the league being outliers with many possession-based goals. Lower-ranked teams also tend to score more on penalties. 

Lower and higher ranked teams tend to pass the ball forward the most, except FCK, who passes the ball backwards more than other teams, as well as doing most short passes. It makes sense that the teams in the middle of the league tend to pass the ball sideways the most, as one would expect them to be more closely matched with their opponents in average, leading to longer time between attacks, which means more sideways passes are needed in order to carry out an attack. It makes sense, as well, that FCK makes many short passes, as a long pass backwards would be detrimental to getting the ball towards the opposition's goal. 

There is a tendency for higher ranked teams to make more shots from inside the box. This leads to the conclusion that FCK, for example, focuses on making short passes towards the opposition's goal, to maximize the amount of shooting chances. FCK has a low goal to chance ratio, and while SonderjyskE and FC Midtjylland often score on their chances, the general tendency seems to be that lower ranked teams score on most of their chances. Lower ranked teams create less chances, so it seems strategically sound to maximize the quality of these, instead of the quantity. 

The higher ranked teams also seem to make more headers towards goal, which makes sense as one would expect the players of higher ranked teams to win more duels against the defenders, and thereby having a higher chance of headers.

Generally, SonderjyskE and FC Midtjylland behave somewhat like lower-ranked teams, except with regards to how they score. One is left to wonder if lower ranked teams would be able to climb considerably in the league, just by focusing less on possession based attacks, and more on set plays and quick attacks. 	
\end{document}
