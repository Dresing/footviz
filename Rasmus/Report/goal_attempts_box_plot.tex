%Name in report: "Goal attempts box plot"
\documentclass[Report.tex]{subfiles}

\begin{document}

\begin{figure}
\center
\includegraphics[width=0.8\textwidth]{"goal_attempts_box_plot_with_interval".pdf}
\caption{Box plot of goal attempts in the first 10 minutes, the next 20 minutes and so on. The distance is the distance from the goal on the x-axis on the football field}
\label{Fig:goal_attempts_box_plot}
\end{figure}

To easily see how the goal attempts change throughout a match, the goal attempts of an entire season has been aggregated and then
split into intervals such that it is possible to see the distance from the goal a goal attempt is made during the first 10 minutes, the next
10 minutes and so on. This information has been visualized using a box plot (Figure \ref{Fig:goal_attempts_box_plot}). As described in the theory section, a box plot quickly allows a user to see different things about a data set, and therefore functions as a quick way to propose hypotheses about the data.

\paragraph{Findings\\}
In the box plot it can be seen that the more minutes into a match a goal attempts is made, the more ``concentrated'' the shots are, meaning that it seems like teams try to get closer to the goal to make goal attempts. This can be seen by looking at the IQR of the first box and the last box, where it can be seen that the first box has a large IQR (meaning that the size of the box is big) compared to the IQR of the last box, where the box is smaller. This means that in the first box, half of the observed values are further away from the median than in the last box. This finding is actually the opposite of what we had expected before the visualization was made, as we had made a hypothesis about the goal attempts being closer to the goal, as the players perhaps were more stressed towards the end of the match. Therefore, this visualization has helped us reject the hypothesis, and instead make a new one, which states that goal attempts are more ``concentrated'' during the last parts of a match.
 
\end{document}