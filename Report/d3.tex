\documentclass[Report.tex]{subfiles}
\usepackage{hyperref}
\begin{document}
	D3, short for Data-Driven Documents, is a open-source JavaScript library that allows DOM (Document Object Model) manipulation based on data.\\
	It is designed to work on most \textit{"modern"} internet browsers, anything newer than Internet Explorer 8.\footnote{\href{https://github.com/d3/d3/wiki\#browser--platform-support}{D3 Wiki on github (link)}}\\
	
	Like R, the code written using D3 will provide high code reusability, as different visualizations will likely include similar elements. Once a visualization is made, it can be kept up to date simply by updating the data files, or not necessary at all if data is gathered through an API.\\ 
	D3 assists with binding arbitrary data to selected DOM elements.\\ Upon using the \texttt{data.enter()} on a selection, any DOM elements in that selection will get one element from \textbf{data} assigned to it, and if there are more elements in \textbf{data}, than in the DOM selection, elements are created for each remaining element in \textbf{data}. If on the other hand, there are more DOM elements in the selection, than in the \textbf{data}, the \texttt{data.exit()} method combined with \texttt{.remove()}, will remove excess elements.\\
	Although D3 can be used to create HTML tables etc. the most common use of D3 for visualization purposes, is through SVG (Scalable Vector Graphics).
	SVG is based on XML, and can be modified / styled on the go through JavaScript for editing the SVG elements, and CSS for styling them. The CSS styling can be set and/or modified at runtime through JavaScript as well. Because of this, the programmer have the possibility to create visualizations that react to what the user is doing in interesting ways, or create dynamic visualizations such as time lapse bubble-charts.\\
	Another reason to use SVG over raster image formats, is that it will look as intended no matter how much it is zoomed / scaled, where raster images like png or jpeg become pixelated when scaled.
	One of the major features in d3 is the Scales sub-library, which provides functionality that can transform all data given to fit inside the given range, while still maintaining the relative position / size between elements.
	This is especially useful when designing responsive web pages, where dimensions of the SVG element may differ.\\
	D3's SVG sub-library brings generators for several non-standard shapes in SVG. These generators can generate SVG paths which resemble the shapes. Some of these shapes are: lines, arcs and chords. These shapes be modified with options such as interpolation, which can be used to smoothen lines in a line chart, while they are still styleable through CSS.\\
	
	While this way of doing visualizations in a lot of cases is preferred due to reusability, and user interactivity, working with large datasets can give unwanted results. 
	If the user have to load more than a few megabytes of data in order to create the visualization, it is going to take some time, and if that data results in a large amount of SVG elements, a dynamic or interactive visualization could become unresponsive, which is undesired, as users are known to be impatient.
\end{document}